\documentstyle[12pt]{article}
\usepackage{html}
\begin{document}
{\Large \bf Independent Component Analysis}

This introduction is based on the
\htmladdnormallink{tutorial paper}{http://www.cis.hut.fi/aapo/papers/IJCNN99_tutorialweb/} by \htmladdnormallink{Aapo Hyv�rinen}{http://www.cs.helsinki.fi/u/ahyvarin/} and \htmladdnormallink{Erkki Oja}{http://www.cis.hut.fi/oja/}.

\vskip 0.4in
\section*{Formulation of the Problem}

A typical example is the ``cocktail party problem''. (See a
\htmladdnormallink{demo}{http://www.cis.hut.fi/projects/ica/icademo/}.) Given
the signals $x_j(t)$ from $m$ microphones recording $n$ speakers in the
room ($m\ge n$), one wants to recover the voice $s_i(t)$ of each speaker.
The problem can be formulated as
\begin{itemize}
\item {\bf Given}
\[	x_i(t)=\sum_{j=1}^n a_{ij} s_j(t)\;\;\;\;\;(i=1,\cdots,m) \]
or in matrix form
\[	\left[ \begin{array}{c} x_1 \\ \cdots \\ x_m \end{array} \right]=
	\left[ \begin{array}{ccc} a_{11} & \cdots & a_{1n} \\
	\cdots & \cdots & \cdots \\ a_{n1} & \cdots & a_{mn}
\end{array} \right]
\left[ \begin{array}{c} s_1 \\ \cdots \\ s_n \end{array} \right]
\;\;\;\;\;\;\;\mbox{or}\;\;\;\;\;\;\;	{\mathbf x=As}	\]
\item {\bf Find}
\begin{itemize}
\item the estimation ${\mathbf y=Wx}$ of the source variables $s_j(t)$
	{\em the independent components},
\item and the linear combination matrix ${\mathbf A}$.
\end{itemize}
\end{itemize}
This is a blind source separation (BSS) problem, i.e., to separate
linearly mixed source signals. The word ``blind'' means that we do
not assume any prior knowledge about sources ${\mathbf s}$ or the mixing
process ${\mathbf A}$ except that the source signals $s_i$ are statistically
independent.

Although this BSS problem seems severely under constrained, the independent
component analysis (ICA) can find nearly unique solutions satisfying
certain properties.

ICA can be compared with
\htmladdnormallink{{\em principal component analysis} (PCA)}{../pca/index.html}
for decorrelation.  Given a set of variables ${\mathbf x}$, PCA finds a
matrix ${\mathbf W}$ so that the components of ${\mathbf y=Wx}$ are
uncorrelated. Only under the special case when ${\mathbf y}=[y_1,\cdots,y_n]$
are gaussian, are they also independent. In comparison, ICA is a more powerful
method in the senese that it satisfies a stronger requirement of finding
${\mathbf W}$ so that the components of ${\mathbf y=Wx}$ are independent
(and therefore are also necessarily uncorrelated).

\section*{Methods of ICA Estimations}

\subsection*{Non-Gaussianity is Independence}

The theoretical foundation of ICA is the
\htmladdnormallink{\em Central Limit Theorem}{http://en.wikipedia.org/wiki/Central_limit_theorem},
which states that the distribution of the sum (average or linear
combination) of $N$ independent random variables approaches Gaussian
as $N\rightarrow \infty$. For example, the face value of a dice has a
uniform distribution from 1 to 6. But the distribution of the sum of a
pair of dice is no longer uniform. It has a maximum probability at the
mean of 7. The distribution of the sum of the face values will be better
approximated by a Gaussian as the number of dice increases.

Specifically, if $x_i$ are random variables independently drawn from an
arbitrary distribution with mean $\mu$ and variance $\sigma^2$. Then the
distribution of the mean $x=\sum_{i=1}^N x_i/N$ approaches Gaussian with
mean $\mu$ and variance $\sigma^2/N$.

To solve the BSS problem, we want to find a matrix ${\mathbf W}$ so that
${\mathbf y=Wx=WAs \approx s}$ is as close to the independent sources
${\mathbf s}$ as possible. This can be seen as the reverse process of the
central limit theorem above.

Consider one component $y_i={\mathbf w_i^TAs}$ of ${\mathbf y}$, where
${\mathbf w_i^T}$ is the ith row of ${\mathbf W}$. As a linear combination
of all components of ${\mathbf s}$, $y_i$ is necessarily more Gaussian
than any of the components unless it is equal to one of them (i.e.,
${\mathbf w_i^TA}$ has only one non-zero component.
In other words, the goal ${\mathbf y \approx s}$ can be achieved by finding
${\mathbf W}$ that maximizes the {\bf non-Gaussianity} of
${\mathbf y=Wx=WAs}$ (so that ${\mathbf y}$ is least Gaussian). This is
the essence of all ICA methods. Obviously if all source variables are
Gaussian, the ICA method will not work.

Based on the above discussion, we get requirements and constrains for the
ICA methods:

\begin{itemize}
\item The number of observed variables $m$ must be no fewer than the number
	of independent sources $n$ (assume $m=n$ in the following).
\item The source components are stochastically independent, and have to be
	non-Gaussian (with possible exception of at most one Gaussian).
\item The estimation of $A$ and $S$ is up to a scaling factor $c_i$.
	Let ${\mathbf C}=diag(c_1,\cdots,c_n)$ and ${\mathbf C}^{-1}=
	diag(1/c_1,\cdots,1/c_n)$,
	and ${\mathbf A}'={\mathbf AC}^{-1}$ and ${\mathbf s'=Cs}$, we have
\[	{\mathbf x=As=[AC^{-1}][Cs]=A's'}	\]
	Also the scaling factor $c_i$ could be either positive or negative.
	For this reason, we will always assume the independent components
	have unit variance $E\{s_i^2\}=1$. As they are also uncorrelated
	(all independent variables are uncorreclated), we have
	$E\{s_is_j\}=\delta_{ij}$, i.e.,
\[	E\{SS^T\}=I	\]
\item The estimated independent components are not in any particular
	order. When the order of the corresponding elements in both
	${\mathbf s}$ and ${\mathbf A}$ is rearranged, ${\mathbf x=As}$
	still holds.
\end{itemize}

All ICA methods are based on the same fundamental approach of finding a
matrix ${\mathbf W}$ that maximizes the non-Gaussianity of ${\mathbf s=Wx}$
thereby minimizing the independence of $S$, and they can be formulated as:
\[ {\bf
\mbox{ICA method}=\mbox{objective function}+\mbox{optimization algorithm}
} \]
All ICA methods are an optimization process (always iterative) to find a
matrix ${\mathbf W}$ to maximize some objective function that measures the
degree of non-Gaussianity or independence of the estimated components
${\mathbf s=Wx}$. In the following, we will discuss some common objective
functions.

\subsection*{Measures of Non-Gaussianity}

The ICA method depends on certain measurement of the non-Gaussianity:
\begin{itemize}
\item {\bf Kurtosis}

Kurtosis is defined as the normalized form of the fourth central moment
of a distribution:
\[	kurt(x)=E\{x^4\}-3(E\{x^2\})^2	\]
If we assume $x$ to have zero mean $\mu_x=E\{x\}=0$ and unit variance
$\sigma^2_x=E\{x^2\}-\mu_x^2=1$, then $E\{x^2\}=1$ and $kurt(x)=E\{x^4\}-3$.
Kurtosis measures the degree of peakedness (spikiness) of a distribution
and it is zero only for Gaussian distribution. Any other distribution's
kurtosis is either positive if it is supergaussian (spikier than Gaussian)
or negative if it is subgaussian (flatter than Gaussian). Therefore the
absolute value of the kurtosis or kurtosis squared can be used to measure
the non-Gaussianity of a distribution. However, kurtosis is very sensitive
to outliers, and it is not a robust measurement of non-Gaussianity.

\htmladdimg{../figures/kurtosis.gif}

\item {\bf Differential Entropy -- Negentropy}

The entropy of a random variable $y$ with density function $p(y)$ is
defined as
\[	H(y)=-\int_{-\infty}^\infty p(y) log\; p(y) dy
	=-E\{ log \; p_i(y) \}	\]
An important property of Gaussian distribution is that it has the maximum
entropy among all distributions over the entire real axis $(-\infty,\infty)$.
(And uniform distribution has the maximum entropy among all distributions
over a finite range.) Based on this property, the differential entropy,
also called {\em negentropy}, is defined as
\[	J(y)=H(y_G)-H(y) \ge 0	\]
where $y_G$ is a Gaussian variable with the same variance as $y$. As $J(y)$
is always greater than zero unless $y$ is Gaussian, it is a good measurement
of non-Gaussianity.

This result can be generalized from random variables to random vectors,
such as ${\mathbf y}=[y_1,\cdots,y_m]^T$, and we want to find a matrix
${\mathbf W}$ so that ${\mathbf y=Wx}$ has the maximum negentropy
$J({\mathbf y})=H({\mathbf y_G)}-H({\mathbf  y})$, i.e., ${\mathbf y}$ is
most non-Gaussian. However, exact $J({\mathbf y})$ is difficult to get as
its calculation requires the specific density distribution function
$p({\mathbf y})$.

\item {\bf Approximations of Negentropy}

The negentropy can be approximated by
\[	J(y)\approx \frac{1}{12}E\{y^3\}^2+\frac{1}{48} kurt(y)^2	\]
However, this approximation also suffers from the non-robustness due to
the kurtosis function. A better approximation is
\[	J(y)\approx \sum_{i=1}^p k_i [ E\{G_i(y)\}-E\{G_i(g)\}]^2 	\]
where $k_i$ are some positive constants, $y$ is assumed to have zero mean
and unit variance, and $g$ is a Gaussian variable also with zero mean and
unit variance. $G_i$ are some non-quadratic functions such as
\[ G_1(y)=\frac{1}{a} log\;cosh\;(a \;y),\;\;\;\;\;G_2(y) =-exp(-y^2/2) \]
where $1 \le a \le 2$ is some suitable constant. Although this approximation
may not be accurate, it is always greater than zero except when $x$ is
Gaussian. In particular, when $p=1$, we have
\[	J(y)=[E\{ G(y)\}-E\{G(g)\}]^2 	\]
Since the second term is a constant, we want to maximize $E\{ G(y) \}$ to
maximize $J(y)$.

\htmladdimg{../figures/G_functions.gif}

\end{itemize}

\subsection*{Minimization of Mutual Information}

The mutual information $I(x,y)$ of two random variables $x$ and $y$ is
defined as
\[	I(x,y)=H(x)+H(y)-H(x,y)=H(x)-H(x|y)=H(y)-H(y|x)		\]
Obviously when $x$ and $y$ are independnent, i.e., $H(y|x)=H(y)$ and
$H(x|y)=H(x)$, their mutual information $I(x,y)$ is zero.

\htmladdimg{../figures/mutual_info.gif}

Similarly the mutual information $I(y_1,\cdots,y_n)$ of a set of $n$
variables $y_i$ ($i=1,\cdots,n$) is defined as
\[	I(y_1,\cdots,y_n)=\sum_{i=1}^n H(y_i)-H(y_1,\cdots,y_n)	\]
If random vector ${\mathbf y}=[y_1,\cdots,y_n]^T$ is a linear transform of
another random vector ${\mathbf x}=[x_1,\cdots,x_n]^T$:
\[	y_i=\sum_{j=1}^n w_{ij} x_j,\;\;\;\;\;\mbox{or}\;\;\;\;{\mathbf y=Wx}	\]
then the entropy of ${\mathbf y}$ is related to that of ${\mathbf x}$ by:
\begin{eqnarray}
H(y_1,\cdots,y_n)&=&H(x_1,\cdots,x_n)+E\;\{ log\;J(x_1,\cdots,x_n)\}
	\nonumber \\
	&=& H(x_1,\cdots,x_n)+ log\;det\; {\mathbf W}
	\nonumber
\end{eqnarray}
where $J(x_1,\cdots,x_n)$ is the Jacobian of the above transformation:
\[	J(x_1,\cdots,x_n)=\left| \begin{array}{ccc}
\frac{\partial y_1}{\partial x_1} & \cdots &\frac{\partial y_1}{\partial x_n} \\
\cdots & \cdots & \cdots \\
\frac{\partial y_n}{\partial x_1} & \cdots &\frac{\partial y_n}{\partial x_n}
\end{array} \right|
=det\;{\mathbf W}	\]

The mutual information above can be written as
\begin{eqnarray}
I(y_1,\cdots,y_n)&=&\sum_{i=1}^n H(y_i)-H(y_1,\cdots,y_n)
	\nonumber \\
	&=&\sum_{i=1}^n H(y_i)-H(x_1,\cdots,x_n)-log\;det\; {\mathbf W}
	\nonumber
\end{eqnarray}
We further assume $y_i$ to be uncorrelated and of unit variance, i.e., the
covariance matrix of ${\mathbf y}$ is
\[
E\{{\mathbf yy^T}\}={\mathbf W}E\{{\mathbf xx^T}\}{\mathbf W^T}={\mathbf I}
\]
and its determinant is
\[	det\;{\mathbf I}=1=(det\;{\mathbf W})\;(det\;E\{{\mathbf xx}^T\})
	\;(det\;{\mathbf W}^T)	\]
This means $det\;{\mathbf W}$ is a constant (same for any ${\mathbf W}$). Also,
as the second term in the mutual information expression $H(x_1,\cdots,x_n)$ is
also a constant (invariant with respect to ${\mathbf W}$), we have
\[	I(y_1,\cdots,y_n)=\sum_{i=1}^n H(y_i)+\mbox{Constant}	\]
i.e., minimization of mutual information $I(y_1,\cdots,y_n)$ is achieved by
minimizing the entropies
\[	H(y_i)=-\int p_i(y_i) log\;p_i(y_i) dy_i=-E\{ log\;p_i(y_i) \}	\]
As Gaussian density has maximal entropy, minimizing entropy is equivalent to
minimizing Gaussianity.

Moreover, since all $y_i$ have the same unit variance, their negentropy becomes
\[	J(y_i)=H(y_G)-H(y_i)=C-H(y_i)	\]
where $C=H(y_G)$ is the entropy of a Gaussian with unit variance, same for
all $y_i$. Substituting $H(y_i)=C-J(y_i)$ into the expression of mutual
information, and realizing the other two terms $H({\mathbf x})$ and
$log\;det\;{\mathbf W}$ are both constant (same for any ${\mathbf W}$), we get
\[	I(y_1,\cdots,y_n)=Const-\sum_{i=1}^n J(y_i)	\]
where $Const$ is a constant (including all terms $C$, $H({\mathbf x})$ and
$log\;det\;{\mathbf W}$)
which is the same for any linear transform matrix $W$. This is the fundamental
relation between mutual information and negentropy of the variables $y_1$. If
the mutual information of a set of variables is decreased (indicating the
variables are less dependent) then the negentropy will be increased, and $y_i$
are less Gaussian. We want to find a linear transform matrix $W$ to minimize
mutual information $I(y_1,\cdots,y_n)$, or, equivalently, to maximize negentropy
(under the assumption that $y_i$ are uncorrelated).

\section*{Preprocessing for ICA}

To simplify the ICA algorithms, the following preprocessing steps are usually
taken:
\begin{itemize}
\item {\bf Centering}

Subtract the mean $E\{{\mathbf x}\}$ from the observed variable ${\mathbf x=As}$
so it has zero mean. By doing so, the sources ${\mathbf s}$ also become zero
mean because $E\{{\mathbf x}\}={\mathbf A}\;E\{{\mathbf s}\}=0$. When the
mixing matrix ${\mathbf A}$ is available, $E\{{\mathbf s}\}$ can be estimated
to be ${\mathbf A}^{-1}E\{{\mathbf x}\}$.

\item {\bf Whitening}

Transform observed variables $X$ so that they are uncorrelated and have unit
variance. We first obtain the eigenvalues $\lambda_i$ and their corresponding
eigenvectors ${\mathbf \phi}_i$ of the covariance matrix $E\{{\mathbf xx}^T\}$,
and form the diagonal eigenvalue matrix ${\mathbf \Lambda}=diag(\lambda_1,
\cdots,\lambda_m)$ and orthogonal eigenvector matrix ${\mathbf \Phi}=
[{\mathbf \phi}_1,\cdots,{\mathbf \phi}_m]$ (${\mathbf \Phi}^{-1}={\mathbf \Phi}^T$).
We have
\[	E\{{\mathbf xx}^T\}{\mathbf \Phi}={\mathbf \Phi\Lambda},
	\;\;\;\mbox{i.e.,}\;\;\;\;
	{\mathbf \Lambda}^{-1/2}{\mathbf \Phi}^T E\{{\mathbf xx}^T\}
	{\mathbf \Phi \Lambda^{-1/2}}={\mathbf I}	\]
If we carry out a linear transform ${\mathbf T}={\mathbf \Lambda}^{-1/2}
{\mathbf \Phi}^T$ so that ${\mathbf x'=TX}=({\mathbf \Lambda}^{-1/2}{\mathbf \Phi}^T) X$, the covariance matrix of ${\mathbf x}'$ becomes
\begin{eqnarray}
E\{{\mathbf x'x'}^T\}&=&E\{({\mathbf \Lambda}^{-1/2}{\mathbf \Phi}^T {\mathbf x})
	({\mathbf \Lambda}^{-1/2}{\mathbf \Phi}^T {\mathbf x})^T\}
	=E\{{\mathbf \Lambda}^{-1/2}{\mathbf \Phi}^T {\mathbf xx}^T
	{\mathbf \Phi \Lambda}^{-1/2}\}
	\nonumber \\
	&=&{\mathbf \Lambda}^{-1/2}{\mathbf \Phi}^T E\{{\mathbf xx}^T\}
	{\mathbf \Phi \Lambda}^{-1/2}={\mathbf I }
	\nonumber
\end{eqnarray}
After the transform ${\mathbf T}={\mathbf \Lambda}^{-1/2}{\mathbf \Phi}^T$,
the mixing process becomes
\[	{\mathbf x'=Tx=TAs=A's}	\]
Here the new mixing matrix ${\mathbf A'=TA}$ is orthogonal, as it satisfies:
\[	E\{{\mathbf x'x'}^T\}={\mathbf I}={\mathbf A}'E\{{\mathbf ss}^T\}
	{\mathbf A'}^T={\mathbf A'A'}^T	\]
(recall that we assume $E\{{\mathbf ss}^T\}={\mathbf I}$). Similarly we also
know that the matrix ${\mathbf W}$ we seek is also orthogonal, as
\[	E\{{\mathbf ss}^T\}={\mathbf I}={\mathbf W}E\{{\mathbf x'x'}^T\}
	{\mathbf W}^T={\mathbf WW}^T	\]

The whitening process reduces the independent variables, the $n^2$ components
of the mixing matrix $A'$ to half ($n(n-1)/2$) due to the constraint that
${\mathbf A}'$ is orthogonal. Moreover, the whitening can also reduce the
dimensionality of the problem by ignoring the components corresponding to
very small eigenvalues (PCA).

\end{itemize}

\section*{FastICA Algorithm}

Summarizing the objective functions discussed above, we see a common goal of
maximizing a function $\sum_i E\{ G(y_i) \}$, where $y_i={\mathbf w}_i^T
{\mathbf x}$ is a component of ${\mathbf y}={\mathbf Wx}$
\[	\sum_i E\{ G(y_i) \}=\sum_i E\{ G( {\mathbf w}_i^T {\mathbf x} ) \}	\]
where ${\mathbf w}_i^T$ is the ith row vector in matrix ${\mathbf W}$. We first
consider one particular component (with the subscript i dropped). This is an
optimization problem which can be solved by Lagrange multiplier method with
the objective function
\[	O({\mathbf w})=E\{ G( {\mathbf w}^T {\mathbf x} ) \}
	-\beta( {\mathbf w}^T{\mathbf w}-1)/2	\]
The second term is the constraint representing the fact that the rows and
columns of the orthogonal matrix ${\mathbf W}$ are normalized, i.e.,
${\mathbf w}^T{\mathbf w}=1$. We set the derivative of $O({\mathbf w})$ with
respect to ${\mathbf w}$ to zero and get
\[	F({\mathbf w})\stackrel{\triangle}{=}
	\frac{\partial O({\mathbf w})}{ \partial {\mathbf w}}
	=E\{ {\mathbf x}g( {\mathbf w}^T {\mathbf x} ) \}-\beta {\mathbf w}=0	\]
where $g(z)=dG(z)/dz$ is the derivative of function $G(z)$. This algebraic
equation system can be solved iteratively by Newton-Raphson method:
\[ {\mathbf w} \Leftarrow {\mathbf w}-J_F^{-1}({\mathbf w}) F({\mathbf w}) \]
where $J_F({\mathbf w})$ is the Jacobian of function $F({\mathbf w})$:
\[
J_F({\mathbf w})=\frac{\partial F}{\partial {\mathbf w}}=
E\{{\mathbf xx}^T g'({\mathbf w}^T{\mathbf x})\}-\beta{\mathbf I}
\]
The first term on the right can be approximated as
\[ E\{{\mathbf xx}^T g'({\mathbf w}^T{\mathbf x})\}
	\approx E\{{\mathbf xx}^T\} E\{g'({\mathbf w}^T{\mathbf x})\}
	=E\{g'({\mathbf w}^T{\mathbf x})\} {\mathbf I}
\]
and the Jacobian becomes diagonal
\[ J_F({\mathbf w})=[E\{g'({\mathbf w}^T{\mathbf x})\}-\beta] {\mathbf I} \]
and the Newton-Raphson iteration becomes:
\[ {\mathbf w} \Leftarrow {\mathbf w}-\frac{1}{E\{g'({\mathbf w}^T{\mathbf x})\}-\beta}[E\{ {\mathbf x}g( {\mathbf w}^T {\mathbf x} ) \}-\beta {\mathbf w}]
	\]
Multiplying both sides by the scaler $\beta-E\{g'({\mathbf w}^T{\mathbf x})\}$,
we get
\[ {\mathbf w} \Leftarrow E\{ {\mathbf x}g( {\mathbf w}^T {\mathbf x} ) \}
-E\{g'({\mathbf w}^T{\mathbf x})\} {\mathbf w}	\]
Note that we still use the same representation ${\mathbf w}$ for the
left-hand side, while its value is actually multiplied by a scaler. This is
taken care of by renormalization, as shown in the following FastICA algorithm:
\begin{enumerate}
\item Choose an initial random guess for ${\mathbf w}$
\item Iterate:
\[ {\mathbf w} \Leftarrow E\{ {\mathbf x}g( {\mathbf w}^T {\mathbf x} ) \}
	-E\{g'({\mathbf w}^T{\mathbf x})\} {\mathbf w}	\]
\item Normalize:
	\[ {\mathbf w} \Leftarrow {\mathbf w}/||{\mathbf w}||	\]
\item If not converged, go back to step 2.
\end{enumerate}

This is a \htmladdnormallink{demo}{http://www.cis.hut.fi/projects/ica/icademo/}
of the FastICA algorithm.

\section*{Appendix}

\subsection*{Entropy}

The entropy of a distribution $p_x(x)$ is defined as
\[	H(X)=-\int p(x)\; log\; p(x) dx=E\{ -log\;p(X) \}	\]
Entropy represents the uncertainty of the random variable. Among all
distributions, uniform distribution has maximum entropy over a finite
region $[a,b]$, while Gaussian distribution has maximum entropy over
the entire real axis.

The joint entropy of two random variables $X$ and $Y$ is defined as
\[ H(X,Y)=-\int p(x,y)\;log\;p(x,y) dx\;dy=E\{-log\;p(X,Y)\} \]

The conditional entropy of $X$ given $y$ is
\[	H(X|y)=-\int p(x|y)\;log\;p(x|y) dx=E\{-log\;p(X|Y) \;|\; Y=y\} \]
and the conditional entropy of $X$ given $Y$ is
\begin{eqnarray}
H(X|Y)&=&\int p(y) H(X|y) dy=-\int p(y) \int p(x|y)\;log\;p(x|y) dx dy
	\nonumber \\
	&=&-\int \int p(x,y)\;log\;p(x|y) dx dy
	=E\{E\{-log\;p(X|Y) \;|\; Y\} \}
	\nonumber
\end{eqnarray}

\subsection*{Mutual information}

Mutual information is defined as
\begin{eqnarray}
I(X,Y)&=&H(X)+H(Y)-H(X,Y)
	\nonumber \\
	&=&E\{ -log\;\;p(X)\}+E\{ -log\;\;p(Y)\}+E\{ -log\;\;p(X,Y)\}
	\nonumber \\
	&=&E\{ log\;\frac{p(X,Y)}{p(X)\;p(Y)} \}
	\nonumber
\end{eqnarray}
Mutual information measures the amount of information shared between the
two random variables $X$ and $Y$. Since
\[	p(X,Y)=p(X|Y) \; p(Y)=p(Y|X) \; p(X)	\]
we have
\begin{eqnarray}
I(X,Y)&=& E\{ log\;\frac{p(X,Y)}{p(X)\;p(Y)} \}
	\nonumber \\
	&=& E\{ log\;\frac{p(X|Y)}{p(X)} \}=H(X)-H(X|Y)
	\nonumber \\
	&=& E\{ log\;\frac{p(Y|X)}{p(Y)} \}=H(Y)-H(Y|X)
	\nonumber
\end{eqnarray}

\subsection*{Functions of random variables}

Assume $X$ is a random variable with distribution $p_x(x)$, then its function
$Y=\phi(X)$ is also a random variable. We have $X=\phi^{-1}(Y)=\psi(Y)$
and $dy/dx=\phi'$ and $dx/dy=\psi'=1/\phi'$.

\begin{itemize}
\item Distribution $p_y(y)$ of $Y$ is related to distribution $p_x(x)$ of $X$:

\begin{itemize}
\item If $Y=\phi(X)$ monotonically increases ($\phi'>0$ and $\psi'>0$), then
\[ P_y(y)=P(Y<y)=P(X<x)=\int_{-\infty}^x p_x(x)dx
	=\int_{-\infty}^{\psi(y)} p_x(\psi(y))dx	\]
\[ p_y(y)=dP_y(y)/dy=p_x(\psi(y))\psi'(y)=p_x(x)/\phi'(x)	\]
\item If $Y=\phi(X)$ monotonically decreases ($\phi'<0$ and $\psi'<0$), then
\[ P_y(y)=P(Y<y)=P(X>x)=\int_x^{\infty} p_x(x)dx
	=\int_{\psi(y)}^x p_x(\psi(y))dx	\]
\[ p_y(y)=dP_y(y)/dy=-p_x(\psi(y))\psi'(y)=p_x(\psi(y))|\psi'(y)|=p_x(x)/|\phi'(x)| \]
\end{itemize}
In general, we have
\[	p_y(y)=\sum_i p_x(x_i)/|\phi'(x_i)| \]
where $x_i$ are solutions for equation $y=\phi(x)$.

\item Entropy $H(Y)$ of $Y$ is related to entropy $H(X)$ of
$X=\phi^{-1}(Y)=\psi(Y)$:

\begin{eqnarray}
 H(Y) &=& -\int p_y(y) \; log \; p_y(y) dy
	=-\int \frac{p_x(x)}{|\phi'(x)|} log \; \frac{p_x(x)}{|\phi'(x)|} dy
	\nonumber \\
	&=&-\int p_x(x) \; log \; \frac{p_x(x)}{|\phi'(x)|} dx
	=-\int p_x(x) \; log \; p_x(x) \; dx+\int p_x(x)\; log \; |\phi'(x)| dx
	\nonumber \\
	&=&H(X)+E\; \{log\;|\phi'(X)|\}
	\nonumber	\end{eqnarray}

\end{itemize}
If the inverse function $X=\psi^{-1}(Y)$ is not unique, than
\[	H(Y)<H(X)+E\; \{log\;|\phi'(x)|\}	\]

This result can be generalized to multi-variables. If
\[	Y_i=\phi_i(X_1,\cdots,X_n),\;\;\;\;\;(i=1,\cdots,n) \]
then
\[ H(Y_1,\cdots,Y_n) \le H(X_1,\cdots,X_n)+E\;\{ log\;J(X_1,\cdots,X_n)\} \]
where $J(X_1,\cdots,X_n)$ is the Jacobian of the above transformation:
\[	J(X_1,\cdots,X_n)=\left| \begin{array}{ccc}
\frac{\partial \phi_1}{\partial X_1} & \cdots &\frac{\partial \phi_1}{\partial X_n} \\
\cdots & \cdots & \cdots \\
\frac{\partial \phi_n}{\partial X_1} & \cdots &\frac{\partial \phi_n}{\partial X_n}
\end{array} \right| \]
In particular, if the functions are linear
\[	Y_i=\sum_{j=1}^n a_{ij} X_j,\;\;\;\;\;(i=1,\cdots,n) \]
then
\[ H(Y_1,\cdots,Y_n) \le H(X_1,\cdots,X_n)+log\;det(A)	\]
where $det(A)$ is the determinant of the transform matrix
$A=[a_{ij}]_{n\times n}$. Again, the equation holds if the transform is unique.


\subsection*{Newton-Raphson Method (Uni-Variate)}

To solve an algebraic equation $f(x)=0$, select a random initial guess $x_0$
and follow the iteration:
\[	x \Leftarrow x-\frac{f(x)}{f'(x)}	\]
This Newton-Raphson formula can be derived below. The equation of the tangent
of $f(x)$ at $x=x_0$ is
\[	f'(x_0) = \frac{f(x_1)-f(x_0)}{x_1-x_0}	\]
If we let $f(x_1)=0$, i.e., $x_1$ is the zero crossing of the tangent, we get
\[	x_1=x_0-\frac{f(x_0)}{f'(x_0)}	\]
which is closer to the desired solution than $x_0$. Repeating the process we
will get $x_2, x_3, \cdots, $ which approach the actual solution.

\htmladdimg{../figures/newtonraphson.gif}

\subsection*{Newton-Raphson Method (Multi-Variate)}

The above method can be generalized to multi-variate case to solve n simultaneous
algebraic equations
\[ \left\{ \begin{array}{c} f_1(x_1,\cdots,x_n)=f_1({\mathbf x})=0	\\
		\cdots \cdots \cdots	\\
		f_n(x_1,\cdots,x_n)=f_n({\mathbf x})=0	\end{array} \right.	\]
where ${\mathbf x}=[x_1,\cdots,x_n]^T$ is an n-dimensional vector. This equation
system can be more concisely represented in vector form as
${\mathbf f}({\mathbf x})=0$. The Newton-Raphson formula for multi-variate
problem is
\[ {\mathbf x} \Leftarrow {\mathbf x}-J_f^{-1}({\mathbf x}) f({\mathbf x}) \]
where $J_f({\mathbf x})$ is the Jacobian of function ${\mathbf f}({\mathbf x})$:
\[	J_f( {\mathbf x})=\left[ \begin{array}{ccc}
\frac{\partial f_1}{\partial x_1} & \cdots & \frac{\partial f_1}{\partial x_n} \\
	\cdots	& \cdots & \cdots 	\\
\frac{\partial f_n}{\partial x_1} & \cdots & \frac{\partial f_n}{\partial x_n}
	\end{array}	\right]	\]

To derive this iteration, consider Taylor series
\[	f_i( {\mathbf x}+\delta {\mathbf x})=
	f_i( {\mathbf x}) +\sum_j \frac{\partial f_i}{\partial x_j} \delta x_j
	+O(\delta {\mathbf x}^2)\;\;\;\;\;(i=1,\cdots,n)
\]
We ignore the terms of $\delta {\mathbf x}^2$ and higher and let
$f_i( {\mathbf x}+\delta {\mathbf x})$ be zero (i.e., ${\mathbf x}+\delta {\mathbf x}$
is the zero-crossing of the tangent),  and get
\[	\sum_j \frac{\partial f_i}{\partial x_j} \delta x_j=-f_i( {\mathbf x})
	\;\;\;\;\;(i=1,\cdots,n) \]
Solving this linear equation system for $\delta x_j$, we get
\[ \delta {\mathbf x}=-J_f^{-1}({\mathbf x}) f({\mathbf x}) \]
and the Newton-Raphson formula:
\[ {\mathbf x} \Leftarrow {\mathbf x}+\delta {\mathbf x}=
{\mathbf x}-J_f^{-1}({\mathbf x}) f({\mathbf x}) \]


\end{document}
