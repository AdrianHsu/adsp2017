\documentclass[12pt, a4paper, onecolumn]{IEEEtran}
\usepackage[utf8]{inputenc}
\usepackage{graphicx}
\usepackage{amsmath}
\usepackage{listings}
\usepackage{xcolor}
\graphicspath{ {img/} }
\title{%
    Independent Component Analysis \\
  \large Term paper ( Tutorial ) Advanced Digital Signal Processing }
\author{Pin-Chun Hsu(B03901023), Prof. Jian-Jiun Ding}
\begin{document}

\maketitle
\section{abstract}
This tutorial presents an introduction to independent component analysis (ICA). Unlike principal component analysis, which is based on the assumptions of uncorrelatedness and  normality,  ICA  is  rooted  in  the  assumption  of  statistical  independence. Foundations and basic knowledge necessary to understand the technique are provided hereafter. Also included is a short  tutorial illustrating the implementation of two ICA algorithms (FastICA and InfoMax) with the use of the Mathematica software.
\section{introduction}

Imagine that you are in a room where two people are speaking simultaneously. You have two microphones, which you hold in different locations. The microphones give you two recorded time signals, which we could denote by $x_1(t)$ and $x_2(t)$, with $x_1$ and $x_2$ the amplitudes, and $t$ the time index. Each of these recorded signals is a weighted sum of the speech signals emitted by the two speakers, which we denote by $s1(t)$ and $s2(t)$. We could express this as a linear equation:
\section{conclusion}
\section{references}


\end{document}
