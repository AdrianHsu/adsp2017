\documentclass[12pt, a4paper, onecolumn]{IEEEtran}
\usepackage[utf8]{inputenc}
\usepackage{graphicx}
\usepackage{amsmath}
\usepackage{listings}
\usepackage{xcolor}
\graphicspath{ {img/} }
\title{%
    Independent Component Analysis \\
  \large Term paper ( Tutorial ) Advanced Digital Signal Processing }
\author{Pin-Chun Hsu(B03901023), Prof. Jian-Jiun Ding}
\begin{document}

\maketitle
\section{abstract}
This tutorial presents an introduction to independent component analysis (ICA). Unlike principal component analysis, which is based on the assumptions of uncorrelatedness and  normality,  ICA  is  rooted  in  the  assumption  of  statistical  independence. Foundations and basic knowledge necessary to understand the technique are provided hereafter. Also included is a short  tutorial illustrating the implementation of two ICA algorithms (FastICA and InfoMax) with the use of the Mathematica software.
\section{introduction}

Imagine that you are in a room where two people are speaking simultaneously. You have two microphones, which you hold in different locations. The microphones give you two recorded time signals, which we could denote by $x_1(t)$ and $x_2(t)$, with $x_1$ and $x_2$ the amplitudes, and $t$ the time index. Each of these recorded signals is a weighted sum of the speech signals emitted by the two speakers, which we denote by $s_1(t)$ and $s_2(t)$. We could express this as a linear equation:
\begin{equation}
x_1(t) = a_{11}s_1 + a_{12}s_2
\end{equation}
\begin{equation}
x_2(t) = a_{21}s_1 + a_{22}s_2
\end{equation}
where $a_{11}$,$a_{12}$,$a_{21}$, and $a_{22}$ are some parameters that depend on the distances of the microphones from the speakers. It would be very useful if you could now estimate the two original speech signals $s_1(t)$ and $s_2(t)$, using only the recorded signals $x_1(t)$ and $x_2(t)$. This is called the cocktail-party problem. For the time being, we omit any time delays or other extra factors from our simplified mixing model.
As an illustration, consider the waveforms in Fig. 1 and Fig. 2. These are, of course, not realistic speech signals, but suffice for this illustration. The original speech signals could look something like those in Fig. 1 and the mixed signals could look like those in Fig. 2. The problem is to recover the data in Fig. 1 using only the data in Fig. 2.
Actually, if we knew the parameters $a_{ij}$, we could solve the linear equation in (1) by classical methods. The point is, however, that if you don’t know the $a_{ij}$, the problem is considerably more difficult.
One approach to solving this problem would be to use some information on the statistical properties of the signals $s_i{t}$ to estimate the aii. Actually, and perhaps surprisingly, it turns out that it is enough to assume that $s_1(t)$ and $s_2(t)$, at each time instant t, are statistically independent. This is not an unrealistic assumption in many cases, and it need not be exactly true in practice. The recently developed technique of Independent Component Analysis, or ICA, can be used to estimate the ai j based on the information of their independence, which allows us to separate the two original source signals $s_1(t)$ and $s_2(t)$ from their mixtures $x_1(t)$ and $x_2(t)$. Fig. 3 gives the two signals estimated by the ICA method. As can be seen, these are very close to the original source signals (their signs are reversed, but this has no significance.)
Independent component analysis was originally developed to deal with problems that are closely related to the cocktail-party problem. Since the recent increase of interest in ICA, it has become clear that this principle has a lot of other interesting applications as well.
Consider, for example, electrical recordings of brain activity as given by an electroencephalogram (EEG). The EEG data consists of recordings of electrical potentials in many different locations on the scalp. These potentials are presumably generated by mixing some underlying components of brain activity. This situation is quite similar to the cocktail-party problem: we would like to find the original components of brain activity, but we can only observe mixtures of the components. ICA can reveal interesting information on brain activity by giving access to its independent components.
Another, very different application of ICA is on feature extraction. A fundamental problem in digital signal processing is to find suitable representations for image, audio or other kind of data for tasks like compression and denoising. Data representations are often based on (discrete) linear transformations. Standard linear transforma- tions widely used in image processing are the Fourier, Haar, cosine transforms etc. Each of them has its own favorable properties (Gonzales and Wintz, 1987).
It would be most useful to estimate the linear transformation from the data itself, in which case the transform could be ideally adapted to the kind of data that is being processed. Figure 4 shows the basis functions obtained by ICA from patches of natural images. Each image window in the set of training images would be a superposition of these windows so that the coefficient in the superposition are independent. Feature extraction by ICA will be explained in more detail later on.
All of the applications described above can actually be formulated in a unified mathematical framework, that of ICA. This is a very general-purpose method of signal processing and data analysis.
\section{conclusion}
\section{references}


\end{document}
